\documentclass[11pt]{report}
\usepackage[spanish]{babel}
\usepackage[utf8]{inputenc}
\usepackage{amsmath}
\usepackage{amssymb}
\usepackage{graphicx}
\graphicspath{ {images/} }

\begin{document}{}

\textbf{54} \textit{The accompanying figure shows the torus that is generated
by revolving the circle}
\[
    (x - a)^2 + z^2 = b^2; (0 < b < a)
\]

\textit{in the xz-plane about the z-axis.}

\begin{itemize}
    \item \textit{Show that this torus can be expressed parametrically as}
    \[
        x = (a + b \cos v) \cos u
    \]
    \[
        y = (a + b \cos v) \sin u
    \]
    \[
        z = b \sin v
    \]
    \textit{where u and v are the parameters shown in the figure
        and} \( 0 \leq u \leq 2 \pi, 0 \leq v \leq 2\pi \)
    \item Use a graphing utility to generate a torus.
\end{itemize}

\begin{figure}[h]
    \centering
    \includegraphics[scale=0.3]{torus.png}
\end{figure}

Sabeemos que la ecuación del toro es:
\[
    (x - a)^2 + z^2 = b^2; (0 < b < a)
\]

Entonces, si consideramos que la parametrización de $z$ para formar un círculo
es \(z = b \sin{v}, 0 \leq v \leq 2 \pi \), entonces:

\begin{equation*}
\begin{split}
    (x - a)^2 + b^2 \sin{v}^2 = b^2 \\
    (x - a)^2 = b^2  - b^2 \sin{v}^2 \\
    (x - a)^2 = b^2 \left( 1 - \sin{v}^2 \right) \\
    (x - a)^2 = b^2 \left( \cos{v}^2 \right) \\
    x - a = b \cos{v} \\
    x = b \cos{v} + a\\
\end{split}
\end{equation*}

Además, como el círculo rota al rededor del eje $y$, entonces:

\[
    x = x \cos{u}
\]
\[
    y = x \sin{u}
\]

Entonces por la igualdad de \(x = b \cos{v} + a\) y por como tomamos a \(z\):

\begin{equation*}
\begin{split}
    x = x \cos{u} &= (a + b \cos{v}) \cos{u} \\
    y = x \sin{u} &= (a + b \cos{v}) \sin{u} \\
    z &= b \sin{v}
\end{split}
\end{equation*}

\begin{figure}[h]
    \centering
    \includegraphics[scale=0.5]{torus_sol.png}
\end{figure}


\newpage

\textbf{37} \\

$\iint \limits_{R} \frac{\sin{x - y}}{\cos{x + y}} \, dA$ \textit{where R is the
triangular region enclosed by the lines} \( y = 0, y = x, x + y = \frac{\pi}{4}\)

Analizando la región \(R\):

\begin{figure}[h]
    \centering
    \includegraphics[scale=0.2]{14_7_37.png}
\end{figure}

\[
    R = \{ x, y | 0 \leq x \leq \frac{\pi}{8}, x \leq y \leq \frac{\pi}{4} - x \}
\]

Notemos que contamos con las funciones:

\[
    y = x
\]
\[
    x + y = \frac{\pi}{4}
\]

Haciendo ambas iguales a algo constante:
\[
    y - x = 0
\]
\[
    x + y = \frac{\pi}{4}
\]

Entonces si tomamos las funciones \( u = u(x, y), v = v(x, y) \):

\[
    u = x + y
\]
\[
    v = y - x
\]

Entonces obtenemos una nueva región \(S\), tal que:

\[
    S = \{x,y | 0 \leq u \leq \frac{\pi}{4}, 0 \leq v \leq u \}
\]

Sumando:

\[
    u + v = (x + y) + (y - x) = y + y + x - x = 2y \rightarrow y = \frac{1}{2}(u + v) = \frac{1}{2} u + \frac{1}{2} v
\]

Restando:

\[
    u - v = (x + y) - (y - x) = x + y - y + x = 2x \rightarrow x = \frac{1}{2}(u - v) = \frac{1}{2} u - \frac{1}{2} v
\]

Entonces buscando el Jacobiano:

\begin{equation*}
\begin{split}
    \frac{\partial(x, y)}{\partial(u, v)}
    &= \left|
        \begin{matrix}
            \frac{\partial x}{\partial u} & \frac{\partial x}{\partial v} \\ \\
            \frac{\partial y}{\partial u} & \frac{\partial y}{\partial v} \\
        \end{matrix}
     \right| \\
     &= \left|
         \begin{matrix}
             \frac{1}{2} & -\frac{1}{2} \\ \\
             \frac{1}{2} & \frac{1}{2} \\
         \end{matrix}
      \right| \\
     &= \frac{1}{4} - \left( -\frac{1}{4} \right) \\
     &= \frac{1}{4} + \frac{1}{4} = \frac{1}{2}
\end{split}
\end{equation*}

Entonces, cambiando la integral:

\begin{equation*}
\begin{split}
    \iint \limits_{R} \frac{\sin{(x - y)}}{\cos{(x + y)}} \, dA
    &= \iint \limits_{S} \frac{\sin{(-v)}}{\cos{(u)}} \frac{\partial(x, y)}{\partial(u, v)} \, dA \\
    &= \frac{1}{2} \iint \limits_{S} \frac{\sin{(-v)}}{\cos{(u)}}  \, dA \\
    &= -\frac{1}{2} \int \limits_{0}^{\frac{\pi}{4}} \int \limits_{0}^{u} \frac{\sin{(v)}}{\cos{(u)}}  \, dv \, du \\
    &= \frac{1}{2} \int \limits_{0}^{\frac{\pi}{4}} \left[ \frac{\cos{(v)}}{\cos{(u)}} \right] \Big |_{0}^{u} \, du \\
    &= \frac{1}{2} \int \limits_{0}^{\frac{\pi}{4}} \left( 1 - \frac{1}{\cos{(u)}} \right) \, du \\
    &= \frac{1}{2} \left[ \frac{\pi}{4} - \int \limits_{0}^{\frac{\pi}{4}} \sec{u} \right] \, du \\
    &= \frac{1}{2} \left[ \frac{\pi}{4} - \ln \left| \sec u + \tan u \right| \Big |_0^{\frac{\pi}{4}} \right]  \\
    &= \frac{1}{2} \left[ \frac{\pi}{4}
        - \ln \left| \sec \left( \frac{\pi}{4} \right) + \tan \left( \frac{\pi}{4} \right) \right|
        + \ln \left| \sec \left( 0 \right) + \tan \left( 0 \right) \right| \right] \\
    &= \frac{1}{2} \left[ \frac{\pi}{4}
        - \ln \left| \sec \left( \frac{\pi}{4} \right) + \tan \left( \frac{\pi}{4} \right) \right| \right] \\
\end{split}
\end{equation*}

\newpage

\textbf{16} The joint density function for $x, y$ is given by

\[
    f(x, y) =
    \begin{cases}
        kxy, $ for $ 0 \leq x \leq y \leq 1 \\
        0, $ otherwise $
    \end{cases}
\]


\textbf{a)} Determine the value of $k$.

Tomemos en cuenta que en una función de densidad conjunta
\[ \int_{-\infty}^{\infty} \int_{-\infty}^{\infty} f(x,y) dxdy = 1\]
Por lo tanto
\[ \int_{0}^{1} \int_{x}^{1} kxy dxdy = 1\]
de donde
\[\frac{1}{8} k = 1 \]
y llegamos a que
\[ k = 8 \]


\textbf{b)} Find the probability that $(x, y)$ lies in the shaded region in Figure 16.57.

\begin{figure}[h]
    \centering
    \includegraphics[scale=0.5]{16.jpeg}
\end{figure}


La probabilidad es
\[ \int_{0}^{1} \int_{x}^{\sqrt{x}} 8xy dxdy = \int_{0}^{1} 4x(y^2) \bigm|_x^{\sqrt{x}} dx \]
\[ = \int_{0}^{1} 4x(x - x^2) dx\]
\[ = 4 \left(\frac{1}{3} x^3 - \frac{1}{4} x^4 \right) \bigm|_0^1 \]
\[ = 4 \left(\frac{1}{3} - \frac{1}{4} \right)	\]
\[ = \frac{1}{3} \]

\newpage

\textbf{23.} If $p_1 (x, y)$ and $p_2 (x, y)$ are joint density functions, then $p_1 (x, y) + p_2 (x, y)$ is a joint density function. \\

En caso de tener $p_1$ y $p_2$ de la siguiente manera:
\[ \int_{-\infty}^{\infty} \int_{-\infty}^{\infty} p_1 (x,y) dxdy\]
\[ \int_{-\infty}^{\infty} \int_{-\infty}^{\infty} p_2 (x,y) dxdy\]
Y que ambos sean igual a 1, al sumarlos
\[ \int_{-\infty}^{\infty} \int_{-\infty}^{\infty} \left[p_1 (x,y) + p_2 (x,y) \right] dxdy =\]
\[\int_{-\infty}^{\infty} \int_{-\infty}^{\infty} p_1 (x,y) dxdy + \int_{-\infty}^{\infty} \int_{-\infty}^{\infty} p_2 (x,y) dxdy  \]
\[ = 1 + 1\]
\[ = 2 \]
Lo cual no puede ser cierto, porque una probabilidad no puede ser mayor a 1.

\newpage

\textbf{24.} If $p(w, h)$ is the probability density function of the weight and height of mothers discussed in Section 16.6, then the probability that a mother weighs $60$ kg and has a height of $170$ cm is $p(60, 170)$. \\

La $p(60, 70)$ no puede ser la probabilidad, más bien (por la definición del Hughes Hallet) lo que está bien es que sea
\[p(60, 170) \Delta w \Delta h\]

\end{document}
